\documentclass[handout]{beamer}
% \documentclass{beamer}
\usepackage{lmodern}
\usepackage[utf8]{inputenc}
\usepackage[english]{babel}

\usepackage{amsmath, amsthm, amsfonts, amssymb}
\usepackage{mathrsfs}

\usepackage[sanserif]{complexity}

% Algorithms
\usepackage[ruled]{algorithm2e}
\SetKw{Continue}{continue}

\usetheme{Madrid}
\usecolortheme{default}

\title[The recompression technique for WE]{The recompression technique for word equations}
\author{Alexander Mayorov}
\institute[RPTU]{RPTU Kaiserslautern--Landau}

\def\Mstruct{\mathcal{M}}

\def\N{\mathbb{N}_{\ge 0}}
\def\Npos{\mathbb{N}_{\ge 1}}
\def\Nsimpl{\mathbb{N}}
\def\R{\mathbb{R}}
\def\Z{\mathbb{Z}}
\def\Q{\mathbb{Q}}
\def\P{\mathcal{P}}

\newlang{\EUF}{EUF}

\DeclareMathOperator{\Disjuncts}{DNF}
\DeclareMathOperator{\DisjunctsOf}{DisjOf}

\def\DisjPhi{\Disjuncts(\varphi)}
\def\DisjPhiPrime{\Disjuncts(\varphi')}
\def\DisjNegPhi{\Disjuncts(\neg\varphi)}
\def\DisjTrue{\Disjuncts(\top)}

\DeclareMathOperator{\CDCL}{CDCL}
\def\CDCLT{\CDCL(\mathrm{T})}

\begin{document}
	
\frame{\titlepage}

\begin{frame}
\frametitle{The problem}

Fix a first-order structure $ \Mstruct $ and consider an $ \Mstruct $-suitable formula $ \varphi(x_1, \dots, x_n) $. Let $ \Pi $ be a partition of $ \{x_1, \dots, x_n\} $.

\begin{definition}[$ \Pi $-decomposable]
	We say that a formula $ \varphi $ \textit{respects} $ \Pi $, if $ \varphi $ is a Boolean combination of formulae each having its free variables within some element of $ \Pi $.
	
	A formula $ \varphi $ is said to be $ \Pi $-decomposable, if there exists an equivalent $ \Pi $-respecting formula (a.k.a. $ \Pi $-decomposition).
\end{definition}

\begin{example}[$ \Mstruct = (\Q, +, <, =, 0, 1) $]
	Let $ \Pi := \{\{x\}, \{y\}\} $. The formula $ \varphi := x + y = 2 \wedge x = 1 $ is $ \Pi $-decomposable since $ \varphi \equiv x = 1 \wedge y = 1 $, whereas $ \varphi := x = y $ is not.
\end{example}

\begin{example}[$ \Mstruct = \EUF $]
	The formula $ \varphi := g(h(x)) = f(y) \wedge h(x) = f(c) $ is $ \Pi $-decomposable for $ \Pi := \{\{x\}, \{y\}\} $, because $ \varphi \equiv g(f(c)) = f(y) \wedge h(x) = f(c) $.
\end{example}

\end{frame}

\begin{frame}

\begin{problem}[The variable decomposition problem]
	Given a formula $ \varphi(x_1, \dots, x_n) $ and a partition $ \Pi $ of $ \{x_1, \dots, x_n\} $, decide whether $ \varphi $ is $ \Pi $-decomposable. If yes, compute a possible $ \Pi $-decomposition.
\end{problem}

\begin{problem}[The variable independence problem]
	Given a formula $ \varphi(x_1, \dots, x_n) $ and variables $ x_i, x_j $, decide whether there exists a partition $ \Pi $ with $ x_i $ and $ x_j $ being in different blocks of $ \Pi $, such that $ \varphi $ is $ \Pi $-decomposable (a.k.a. $ x_i $ and $ x_j $ are independent).
\end{problem}

\begin{example}[$ \Mstruct = (\Q, +, <, =, 0, 1) $]
	Let $ \Pi := \{\{x\}, \{y\}\} $. In $ \varphi := x + y = 2 \wedge x = 1 $, the variables $ x $ and $ y $ are independent. This is not the case for $ \varphi := x = y $.
\end{example}

Both problems are known to be \alert{undecidable} in general \cite{libkin:2003}.

\vspace{2pt}

$ \rightsquigarrow $ we thus focus only on quantifier-free fragments of FOL.

\end{frame}

\begin{frame}
	\frametitle{Nature of the problem and my main result}
	
	Intuitively, deciding $ \Pi $-decomposability is a hard problem because it is hard to witness the fact that some formula is not $ \Pi $-decomposable, even for fixed first-order theories.
	
	\vspace{5pt}
	
	All such known witnesses work either only for ``discrete'' fragments of first-order logic, or they rely on very general techniques which yield decision procedures whose complexity is either high in theory, or infeasible in practice, or both \cite{libkin:2000, libkin:2003, hague:2020, markgraf:2021, veanes:2017}.
	
	\vspace{5pt}
	
	\textbf{My main result}: a novel technique for solving the variable decomposability problem over quantifier-free first-order theories satisfying certain convexity assumptions. This technique yields an efficient decision procedure, which moreover can compute a possible $ \Pi $-decomposition if one exists.
\end{frame}

\begin{frame}
	\frametitle{First step: reduce to another problem}
	
	\begin{problem}[The covering problem]
		Given a formula $ \varphi $, a binary partition $ \Pi $ and a predicate set $ \Gamma \in \DisjPhi $, compute a $ \Pi $-respecting formula $ \psi $ such that \begin{itemize}
			\item $ \Gamma \models \psi $
			\item $ \psi \models \varphi $ if $ \varphi $ is $ \Pi $-decomposable
		\end{itemize}
	\end{problem}
	
	\begin{theorem}
		Suppose the conjunctive fragment of the theory of $ \Mstruct $ is decidable in polynomial time. Then there exists a co-nondeterministic $ O(2^{O(n)} \cdot \poly(n + m)) $-time Turing reduction from the variable decomposition problem to the covering problem, where $ n $ is the size of $ \varphi $ and $ m $ is the maximum possible size of the $ \psi $ formula produced by the oracle for the covering problem.
	\end{theorem}
	
\end{frame}

\begin{frame}
	
	\begin{proof}[Proof sketch]
		For every $ \Gamma \in \DisjPhi $, we call the oracle solving the covering problem and obtain a $ \Pi $-respecting formula $ \psi_\Gamma $ such that $ \Gamma \models \psi_\Gamma $ and $ \varphi $ being $ \Pi $-decomposable implies $ \psi_\Gamma \models \varphi $. If for all $ \Gamma \in \DisjPhi $, $ \psi_\Gamma \models \varphi $, then we return that the formula is $ \Pi $-decomposable and output the $ \Pi $-decomposition \[
		\varphi \equiv \bigvee_{\Gamma \in \DisjPhi} \psi_\Gamma
		\] Otherwise, we reject the formula $ \varphi $ as being not $ \Pi $-decomposable and terminate.
	\end{proof}

	\begin{theorem}
		If the conjunctive fragment of the theory of $ \Mstruct $ is decidable in polynomial time, $ \Mstruct $ satisfies the small model property, and membership of a small model in $ \psi $ is in $ \NP $, then the above reduction yields a $ \coNP $ algorithm for the variable decomposition problem.
	\end{theorem}
	
\end{frame}

\begin{frame}
	\frametitle{Conclusion}
	
	\textbf{Main novel technique}: model flooding
	
	\vspace{10pt}
	
	\textbf{Best known upper bound}: double exponential
	
	\vspace{10pt}
	
	With my approach, we obtain:
	
	\begin{theorem}
		In rational linear arithmetic, the variable decomposition problem is $ \coNP $-hard and in $ \PSPACE $. The same holds for any convex quantifier-free first-order theory.
	\end{theorem}
	
	\vspace{5pt}
	
	The logic has to satisfy only certain \textit{convexity conditions} in order for the approach to work. Moreover, unlike known approaches, my algorithm can be easily integrated into $ \CDCLT $, which makes it also useful in practice.
	
\end{frame}

\begin{frame}[allowframebreaks]
\frametitle{References}
\bibliographystyle{amsalpha}
\bibliography{we_recompression}
\end{frame}

\end{document}