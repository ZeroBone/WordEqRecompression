% \documentclass[handout]{beamer}
\documentclass{beamer}
\usepackage{lmodern}
\usepackage[utf8]{inputenc}
\usepackage[english]{babel}

\usepackage{amsmath, amsthm, amsfonts, amssymb}
\usepackage{mathrsfs}

\usepackage{oubraces}

\usepackage{tikz}
\usetikzlibrary{calc}

\usepackage[sanserif]{complexity}

% Algorithms
\usepackage[ruled]{algorithm2e}
\SetKw{Continue}{continue}

\usetheme{Madrid}
\usecolortheme{default}

\title[The recompression technique for WE]{The recompression technique for word equations}
\author{Alexander Mayorov}
\institute[RPTU]{University of Kaiserslautern--Landau (RPTU)}

% \newclass{\TWOEXP}{2EXPTIME}
\newclass{\THREENEXP}{3NEXPTIME}
\newclass{\FOURNEXP}{4NEXPTIME}
\renewclass{\EXP}{EXPTIME}
\renewclass{\NEXP}{NEXPTIME}

\def\Mstruct{\mathcal{M}}

\def\N{\mathbb{N}_{\ge 0}}
\def\Npos{\mathbb{N}_{\ge 1}}
\def\Nsimpl{\mathbb{N}}
\def\R{\mathbb{R}}
\def\Z{\mathbb{Z}}
\def\Q{\mathbb{Q}}
\def\P{\mathcal{P}}

\def\X{\mathcal{X}}

% expected value
\def\ExpVal{\E}

\newcommand{\var}[1]{\textcolor{blue}{#1}}
\newcommand{\sol}[1]{\textcolor{orange}{#1}}

\definecolor{OliveGreen}{rgb}{0,0.6,0}
\definecolor{LightCyan}{rgb}{0.88,1,1}
\definecolor{magentazb}{RGB}{255,0,255}
\definecolor{LightBlueZB}{RGB}{51,51,179} % #3333B3

\newcommand{\freshVar}[1]{\textcolor{red}{#1}}
\newcommand{\anotherFreshVar}[1]{\textcolor{OliveGreen}{#1}}
\newcommand{\alertGreen}[1]{\textcolor{OliveGreen}{#1}}
\newcommand{\alertBlue}[1]{\textcolor{LightBlueZB}{#1}}

\newcommand{\stepdown}[1]{
	\begin{center}
		\begin{tabular}{c c}
			$ \Downarrow $ & #1
		\end{tabular}
	\end{center}
}

\newcommand{\unknowneqpart}[1]{\sol{\boxed{\mbox{#1}}}}
\newcommand{\redeqbox}[1]{\freshVar{\boxed{\mbox{#1}}}}

\newenvironment<>{openproblem}[1][]{%
	\setbeamercolor{block title}{fg=white,bg=magentazb}%
	\begin{block}#1}{\end{block}}

\AtBeginSection[]
{
	\begin{frame}
		\frametitle{Table of Contents}
		\tableofcontents[currentsection]
	\end{frame}
}

\begin{document}
	
\frame{\titlepage}

\section{Word equations}

\begin{frame}
\frametitle{The problem}

Fix some finite alphabet $ \Gamma $ and set of variables $ \X $.

\begin{problem}[Satisfiability of word equations]
	Given an equation $ U = V $ where $ U, V \in (\Gamma \cup \X)^* $, does there exist a substitution $ \varphi : \X \rightarrow \Gamma^* $ satisfying the equation?
\end{problem}

\begin{example}[Satisfiable equation]
	The equation $ b\var{x}ba\var{y}b = \var{y}bb\var{x}ab $ is \alertGreen{satisfiable} via $ \varphi(\var{x}) := \sol{ab} $, $ \varphi(\var{y}) := \sol{ba} $: \begin{center}
		\begin{tabular}{c c c c c c c c c c}
			$ b $ & $ \var{x} $ & $ ba $ & $ \var{y} $ & $ b $ & $ = $ & $ \var{y} $ & $ bb $ & $ \var{x} $ & $ ab $ \\
			$ b $ & $ \sol{ab} $ & $ ba $ & $ \sol{ba} $ & $ b $ & $ = $ & $ \sol{ba} $ & $ bb $ & $ \sol{ab} $ & $ ab $
		\end{tabular}
	\end{center}
\end{example}

\begin{example}[Unsatisfiable equations]
	The equations $ a\var{x} = b\var{y} $, $ a\var{x}\var{y} = \var{y}a\var{x}a $ and $ \var{x}a = ab\var{x} $ are \alert{unsatisfiable}.
\end{example}

\end{frame}

\begin{frame}
\frametitle{History}

\begin{itemize}
	\item[1977] Word equations are decidable \cite{makanin:1977}.
	\item[1990] Makanin's algorithm \cite{makanin:1977} improved to run in $ \FOURNEXP $ \cite{jaffar:1990, schulz:1990}.
	\item[1996] Further improvement to $ \THREENEXP $ \cite{koscielski_pacholski:1996}.
	\item[1998] Further improvement to $ \EXPSPACE $ \cite{gutierrez:1998}.
	\item[1999] $ \NEXP $ algorithm via a double-exponential upper bound on the size of the length-minimal solution \cite{plandowski:1999}.
	\item[2004] $ \PSPACE $ algorithm via a rewriting system \cite{plandowski:2004}. Soundness trivial, complicated proof of completeness.
	\item[2016] $ \PSPACE $ algorithm via the \textit{recompression technique} \cite{jez:2016}.
	\item[2022] $ \NSPACE(n) $ algorithm via \textit{recompression} and Huffman coding \cite{jez:2022}.
\end{itemize}

\pause

\textbf{Best known lower bound}: $ \NP $-hard

\begin{openproblem}{Long-standing open problem}
	Are word equations in $ \NP $?
\end{openproblem}

\end{frame}

\begin{frame}
	\frametitle{Notation}
	
	\begin{itemize}
		\item Consider the word equation $ ca\var{y}aa\var{x}a = \var{x}aaaca\var{y} $.
		\item We visualize the solution \[
			\varphi(\var{x}) := \sol{cabcba}
			\quad\quad \varphi(\var{y}) := \sol{bcbaa}
		\] succinctly as follows: \[
		\overunderbraces{&&&\br{3}{\var{y}}&&&&\br{3}{\var{x}}}
		{&c\ a&\ &\sol{b\ c\ b\ a}&\ &a&\ &a\ a&\ &c\ a&\ &\sol{b\ c\ b\ a}&\ a}%
		{&\br{3}{\var{x}}&&&&&&&&\br{2}{\var{y}}}
		\]
		\item Over-braces encode the variables appearing on the left-hand side $ ca\var{y}aa\var{x}a $ of the equation, while the under-braces indicate the variables occuring on the right-hand side $ \var{x}aaaca\var{y} $.
	\end{itemize}
	
\end{frame}

\section{Solving word equations via recompression}

\begin{frame}
	\frametitle{Intuition: pair compression}
	
	\begin{columns}
		\column{0.5\textwidth}\textbf{Given equation}\[
		\overunderbraces{&&&\br{3}{\var{y}}&&&&\br{3}{\var{x}}}
		{&c\ a&\ &\unknowneqpart{?\ ?\ ?\ ?}&\ &a&\ &a\ a&\ &c\ a&\ &\unknowneqpart{?\ ?\ ?\ ?}&\ a}%
		{&\br{3}{\var{x}}&&&&&&&&\br{2}{\var{y}}}
		\]
		\column{0.5\textwidth}\textbf{Solved equation}\[
		\overunderbraces{&&&\br{3}{\var{y}}&&&&\br{3}{\var{x}}}
		{&c\ a&\ &\sol{b\ c\ b\ a}&\ &a&\ &a\ a&\ &c\ a&\ &\sol{b\ c\ b\ a}&\ a}%
		{&\br{3}{\var{x}}&&&&&&&&\br{2}{\var{y}}}
		\]
	\end{columns}
	\pause
	\stepdown{Compress $ \sol{ba} \rightsquigarrow \freshVar{d} $}
	\begin{columns}
		\column{0.5\textwidth}\[
		\overunderbraces{&&&\br{3}{\var{y}}&&&&\br{3}{\var{x}}}
		{&c\ a&\ &\unknowneqpart{?\ ?\ ?}&\ &a&\ &a\ a&\ &c\ a&\ &\unknowneqpart{?\ ?\ ?}&\ a}%
		{&\br{3}{\var{x}}&&&&&&&&\br{2}{\var{y}}}
		\]
		\column{0.5\textwidth}\[
		\overunderbraces{&&&\br{3}{\var{y}}&&&&\br{3}{\var{x}}}
		{&c\ a&\ &\sol{b\ c}\ \freshVar{d}&\ &a&\ &a\ a&\ &c\ a&\ &\sol{b\ c}\ \freshVar{d}&\ a}%
		{&\br{3}{\var{x}}&&&&&&&&\br{2}{\var{y}}}
		\]
	\end{columns}
	\pause
	\stepdown{Compress $ ca \rightsquigarrow \anotherFreshVar{e} $}
	\begin{columns}
		\column{0.5\textwidth}\[
		\overunderbraces{&&&\br{3}{\var{y}}&&&&\br{3}{\var{x}}}
		{&\anotherFreshVar{e}&\ &\unknowneqpart{?\ ?\ ?}&\ &a&\ &a\ a&\ &\anotherFreshVar{e}&\ &\unknowneqpart{?\ ?\ ?}&\ a}%
		{&\br{3}{\var{x}}&&&&&&&&\br{2}{\var{y}}}
		\]
		\column{0.5\textwidth}\[
		\overunderbraces{&&&\br{3}{\var{y}}&&&&\br{3}{\var{x}}}
		{&\anotherFreshVar{e}&\ &\sol{b\ c}\ \freshVar{d}&\ &a&\ &a\ a&\ &\anotherFreshVar{e}&\ &\sol{b\ c}\ \freshVar{d}&\ a}%
		{&\br{3}{\var{x}}&&&&&&&&\br{2}{\var{y}}}
		\]
	\end{columns}
\end{frame}

\begin{frame}
	\frametitle{Central idea}
	
	\textbf{Idea}: nondeterministically and iteratively compress the given word equation $ U = V $ in a sound and complete way.
	
	\textbf{One compression step}:
	
	\begin{center}
		\begin{tikzpicture}[
			grow=right,
			edge from parent/.style = {draw, -latex},
			sibling distance        = 2em,
			level distance          = 10em,
			]
			\node {$ U = V $}
			child {node {$ U_k = V_k $}}
			child {node {$ \phantom{U_k}\vdots\phantom{V_k} $}}
			child {node {$ U_3 = V_3 $}}
			child {node {$ U_2 = V_2 $}}
			child {node {$ U_1 = V_1 $}};
		\end{tikzpicture}
	\end{center}
	
	\begin{center}
		\begin{tabular}{r l c l}
			\alertBlue{Completeness}: & $ U = V $ \alertGreen{satisfiable} & $ \Rightarrow $ & $ \exists i : U_i = V_i $ \alertGreen{satisfiable} \\
			\alertBlue{Soundness}: & $ U = V $ \alert{unsatisfiable} & $ \Rightarrow $ & $ \forall i : U_i = V_i $ \alert{unsatisfiable}
		\end{tabular}
	\end{center}
\end{frame}

\begin{frame}
	\frametitle{Naive pair compression is problematic}
	
	Clearly, replacing $ ab $ by a fresh letter $ \freshVar{c} $ is sound. However, this operation is not always complete. Consider \alertGreen{satisfiable} equation \[
	\var{x}bab = ab\var{x}b
	\] with solution \[
	\overunderbraces{&\br{3}{\var{x}}}
	{&a\ b&\ &a&\ &b\ a&\ &b}%
	{&&&\br{3}{\var{x}}}
	\]\begin{columns}
		\column{0.5\textwidth}
		\stepdown{Compress $ ab \rightsquigarrow \freshVar{c} $}\[
		\var{x}b\freshVar{c} = \freshVar{c}\var{x}b
		\]\fbox{
			\parbox{0.9\textwidth}{
				\begin{center}
					\alert{Unsatisfiable}\\(because otherwise $ \varphi(\var{x})b \in c^* $)
				\end{center}
			}
		}
		\column{0.5\textwidth}
		\stepdown{Compress $ ba \rightsquigarrow \freshVar{c} $}\[
		\var{x}\freshVar{c}b = ab\var{x}b
		\]\fbox{
			\parbox{0.9\textwidth}{
				\begin{center}
					\alert{Unsatisfiable}\\(because $ |\var{x}\freshVar{c}b|_{\freshVar{c}} > |ab\var{x}b|_{\freshVar{c}} $)
				\end{center}
			}
		}
	\end{columns}
\end{frame}

\begin{frame}
	\frametitle{Crossing occurrences make compression not complete}
	
	\textbf{Problem}: We cannot replace $ \alertGreen{ab} $ by $ \freshVar{c} $ in \[
	\overunderbraces{&\br{3}{\var{x}}}
	{&a\ b&\ &\alertBlue{a}&\ &\alertBlue{b}\ \alertGreen{a}&\ &\alertGreen{b}}%
	{&&&\br{3}{\var{x}}}
	\] because $ ab $ has two crossing occurrences \alertBlue{ab} and \alertGreen{ab}, i.e., occurrences coming neither from $ U $ or $ V $ nor from $ \varphi(\var{x}) $ for some $ \var{x} $.
	
	\begin{definition}[\cite{jez:2016, jez:2014, jez:2015}]
		The occurrence of a substring $ u \in \Gamma^+ $ in a solution $ \varphi(U) $ to $ U = V $ is called \begin{center}
			\begin{tabular}{r l}
				\alertGreen{explicit} & if it comes from a substring $ u $ of $ U $ or $ V $ \\
				\alertBlue{implicit} & if it comes $ \varphi(\var{x}) $ for some $ \var{x} \in \X $ \\
				\alert{crossing} & otherwise
			\end{tabular}
		\end{center}
	\end{definition}
\end{frame}

\begin{frame}
	\frametitle{Crossing occurrences make compression not complete}
	
	\textbf{Solution}: Before compressing a pair $ ab $, ensure that $ ab $ has no crossing occurrences by ``popping-in'' letters from variables:
	
	\[
	\overunderbraces{&\br{3}{\var{x}}}
	{&a\ b&\ &\alertBlue{a}&\ &\alertBlue{b}\ \alertGreen{a}&\ &\alertGreen{b}}%
	{&&&\br{3}{\var{x}}}
	\]\stepdown{Set $ \var{x} := \var{x}a $}\[
	\overunderbraces{&\br{1}{\var{x}}}
	{&a\ b&\ &\alertBlue{a\ b}&\ &\alertGreen{a\ b}}%
	{&&&\br{1}{\var{x}}}
	\]\stepdown{Compress $ ab \rightsquigarrow \freshVar{c} $}\[
	\overunderbraces{&\br{1}{\var{x}}}
	{&\freshVar{c}&\ &\freshVar{c}&\ &\freshVar{c}}%
	{&&&\br{1}{\var{x}}}
	\]
	
	\begin{center}
		$ ab $ has no crossing occurrences $ \Rightarrow $ \alertBlue{sound} and \alertBlue{complete} transformation
	\end{center}
\end{frame}

\begin{frame}
	\frametitle{Pair compression algorithm}
	
	\begin{algorithm}[H]
		\caption{Pair compression \cite[Algorithm 1]{jez:2016}}
		\SetAlgoLined
		\SetKwFunction{compresspair}{compress\_pair}
		\SetKwProg{compresspairproc}{Function}{}{end}
		\KwIn{Letters $ a, b \in \Gamma $}
		\compresspairproc{\compresspair{$ a $, $ b $}}{
			\nl Replace every $ ab $ appearing in $ U $ or $ V $ by a fresh letter $ \freshVar{c} $\;
		}
	\end{algorithm}
\end{frame}

\begin{frame}
	\frametitle{Popping algorithm}
	
	\begin{algorithm}[H]
		\caption{Popping algorithm \cite[Algorithm 3]{jez:2016}}
		\SetAlgoLined
		\SetKwFunction{pop}{pop}
		\SetKwProg{popproc}{Function}{}{end}
		\KwIn{Blocks $ \Gamma_l $ and $ \Gamma_r $ of a partition $ \Gamma = \Gamma_l \cup \Gamma_r $}
		\popproc{\pop{$ \Gamma_l $, $ \Gamma_r $}}{
			\nl \ForEach{$ \var{x} \in \X $}{
				\nl Let $ b $ be the first letter of $ \varphi(\var{x}) $\;
				\nl \If{$ b \in \Gamma_r $}{
					\nl Replace $ \var{x} $ either by $ b\var{x} $ or by $ b $\;
				}
				\nl Let $ a $ be the last letter of $ \varphi(\var{x}) $\;
				\nl \If{$ a \in \Gamma_l $}{
					\nl Replace $ \var{x} $ either by $ \var{x}a $ or by $ a $\;
				}
			}
		}
	\end{algorithm}
\end{frame}

\begin{frame}
	\frametitle{Block compression}
	
	\begin{itemize}
		\item We have discussed how to compress $ ab \in \Gamma^2 $ for $ a \neq b $.
		\item But what about blocks $ a^k $ for $ k \ge 1 $?
		\item We show that $ a^k $ can be compressed using an analogous approach.
	\end{itemize}

	\vspace{2em}
	
	\begin{center}
		\begin{tabular}{r|l}
			\textbf{Step 1} & \begin{tabular}{@{}l@{}}
				For all occurrences of $ a^k $ in $ \varphi(U) $ for $ k $ maximal,\\
				ensure that $ a^k $ is not crossing.
			\end{tabular}\\\hline
			\textbf{Step 2} & Replace every occurrence of $ a^k $ by a fresh letter $ a_k $.
		\end{tabular}
	\end{center}
\end{frame}

\begin{frame}
	\frametitle{Block compression: intuition}
	
	\only<1-5|handout:1-5>{\[
		\overunderbraces{&&&\br{3}{\var{y}}&&&&\br{3}{\var{x}}}
		{&\sol{c\ a}&\ &\sol{b\ c\ b\ a}&\ &a&\ &a\ a&\ &\sol{c\ a}&\ &\sol{b\ c\ b\ a}&\ a}%
		{&\br{3}{\var{x}}&&&&&&&&\br{2}{\var{y}}}
	\]}\only<1-5|handout:1-5>{\stepdown{Set $ \var{x} := \sol{c}\var{x}\sol{a} $}}\only<2|handout:2>{\[
		\overunderbraces{&&&&&\br{5}{\var{y}}&&&&&&\br{3}{\var{x}}}
		{&\sol{c}&\ &\sol{a}&\ &\sol{b}\ \sol{c}\ \sol{b}&\ &\sol{a}&\ &a&\ &a\ a&\ &\sol{c}&\ &\sol{a}&\ &\sol{b\ c\ b}&\ &\sol{a}&\ &a}%
		{&&&\br{3}{\var{x}}&&&&&&&&&&&&\br{5}{\var{y}}}
		\]
	}\only<3-|handout:3->{\[
		\overunderbraces{&&&&&\br{5}{\var{y}}&&&&&&\br{3}{\var{x}}}
		{&c&\ &a&\ &\sol{b}\ \sol{c}\ \sol{b}&\ &\sol{a}&\ &\sol{a}&\ &a\ a&\ &c&\ &a&\ &\sol{b\ c\ b}&\ &\sol{a}&\ &\sol{a}}%
		{&&&\br{3}{\var{x}}&&&&&&&&&&&&\br{5}{\var{y}}}
		\]
	}\only<4-|handout:4->{\stepdown{Set $ \var{y} := \sol{b}\var{y}\sol{aa} $}}\only<4|handout:4>{\[
		\overunderbraces{&&&&&&&\br{1}{\var{y}}&&&&&&&&&&\br{5}{\var{x}}}
		{&c&\ &a&\ &\sol{b}&\ &\sol{c}\ \sol{b}&\ &\sol{a}&\ &\sol{a}&\ &a\ a&\ &c&\ &a&\ &\sol{b}&\ &\sol{c\ b}&\ &\sol{a}&\ &\sol{a}}%
		{&&&\br{5}{\var{x}}&&&&&&&&&&&&&&\br{1}{\var{y}}}
	\]}\only<5-|handout:5->{\[
	\overunderbraces{&&&&&&&\br{1}{\var{y}}&&&&&&\br{5}{\var{x}}}
	{&c&\ &a&\ &b&\ &c\ b&\ &\redeqbox{a\ a\ a\ a}&\ &c&\ &a&\ &b&\ &c\ b&\ &a&\ &a}%
	{&&&\br{5}{\var{x}}&&\br{1}{\substack{\text{Block is no}\\\text{longer crossing}}}&&&&&&&&\br{1}{\var{y}}}
	\]}\only<6-|handout:6->{\stepdown{Compress $ \redeqbox{a\ a\ a\ a} \rightsquigarrow \freshVar{a_4} $}}\only<6-|handout:6->{\[
	\overunderbraces{&&&&&&&\br{1}{\var{y}}&&&&&&\br{5}{\var{x}}}
	{&c&\ &a&\ &b&\ &c\ b&\ &\freshVar{a_4}&\ &c&\ &a&\ &b&\ &c\ b&\ &a&\ &a}%
	{&&&\br{5}{\var{x}}&&&&&&&&&&\br{1}{\var{y}}}
	\]}
\end{frame}

\begin{frame}
	\frametitle{Block compression: intuition (continued)}
	
	\only<1>{\[
		\overunderbraces{&&&&&&&\br{1}{\var{y}}&&&&&&\br{5}{\var{x}}}
		{&c&\ &a&\ &b&\ &c\ b&\ &\freshVar{a_4}&\ &c&\ &a&\ &b&\ &c\ b&\ &a&\ &a}%
		{&&&\br{5}{\var{x}}&&&&&&&&&&\br{1}{\var{y}}}
		\]
	}\only<2->{\[
		\overunderbraces{&&&&&&&\br{1}{\var{y}}&&&&&&\br{5}{\var{x}}}
		{&c&\ &a&\ &b&\ &c\ b&\ &\freshVar{a_4}&\ &c&\ &a&\ &b&\ &c\ b&\ &\redeqbox{a\ a}}%
		{&&&\br{5}{\var{x}}&&&&&&&&&&\br{1}{\var{y}}}
		\]
	}\only<3->{\stepdown{Compress $ \redeqbox{a\ a} \rightsquigarrow \freshVar{a_2} $}}\only<3->{\[
		\overunderbraces{&&&&&&&\br{1}{\var{y}}&&&&&&\br{5}{\var{x}}}
		{&c&\ &a&\ &b&\ &c\ b&\ &\freshVar{a_4}&\ &c&\ &a&\ &b&\ &c\ b&\ &\freshVar{a_2}}%
		{&&&\br{5}{\var{x}}&&&&&&&&&&\br{1}{\var{y}}}
		\]
	}
	\only<3->{\begin{center}
		This concludes the compression of all $ a $-blocks.
	\end{center}}
\end{frame}

\begin{frame}
	\frametitle{Block compression for non-crossing blocks}
	
	Once all $ a $-blocks of length at least two are non-crossing, we simply replace them by fresh variables:
	
	\vspace{1em}
	
	\begin{algorithm}[H]
		\caption{Block compression \cite[Algorithm 2]{jez:2016}}
		\SetAlgoLined
		\SetKwFunction{compressblock}{compress\_block}
		\SetKwProg{compressblockproc}{Function}{}{end}
		\KwIn{Letter $ a \in \Gamma $}
		\compressblockproc{\compressblock{$ a $}}{
			\nl \ForEach{occurrence of $ a^l $ for some $ l \ge 2 $ in $ U $ or $ V $}{
				\nl Replace this occurrence by $ a_l $, where $ a_l $ is a letter uniquely determined by $ a $ and $ l $\;
			}
		}
	\end{algorithm}
\end{frame}

\begin{frame}
	\frametitle{Making blocks non-crossing}
	
	We ensure that the compressed $ a $-blocks are non-crossing as follows:
	
	\vspace{1em}
	
	\begin{algorithm}[H]
		\caption{Cutting prefixes and suffixes \cite[Algorithm 5]{jez:2016}}
		\SetAlgoLined
		\SetKwFunction{cutprefsuff}{cut\_pref\_suff}
		\SetKwProg{cutprefsuffproc}{Function}{}{end}
		\cutprefsuffproc{\cutprefsuff{}}{
			\nl \ForEach{$ \var{x} \in \X $}{
				\nl Guess $ l \ge 1 $, $ r \ge 0 $, $ a, b \in \Gamma $ such that $ \varphi(\var{x}) = a^lwb^r $ holds for some $ w \in \Gamma^* $ where $ w $ neither starts with $ a $ nor ends with $ b $\;
				\tcp{Use binary encoding to store $ a^l $ and $ b^r $}
				\nl Replace $ \var{x} $ either by $ a^l\var{x}b^r $ or by $ a^lb^r $\;
			}
		}
	\end{algorithm}
\end{frame}

\begin{frame}
	\frametitle{Algorithm for word equations}
	
	\begin{algorithm}[H]
		\caption{Satisfiability of word equations \cite[Algorithm 7]{jez:2016}}
		\SetAlgoLined
		\SetKwFunction{wesat}{we\_sat}
		\SetKwProg{wesatproc}{Function}{}{end}
		\wesatproc{\wesat{}}{
			\nl \While{$ \left|U\right| \ge 2 \vee \left|V\right| \ge 2 $}{
				% \tcp{Block compression}
				\nl \cutprefsuff{}\tcp*[r]{Make sure $ a^l $ is non-crossing}
				\nl \ForEach{$ a \in \Gamma $}{
					\nl \compressblock{$ a $}\tcp*[r]{Compress every $ a $-block}
				}
				% \tcp{Two rounds of pair compression}
				\nl \ForEach{$ i \in \{1, 2\} $}{
					\nl Guess partition of $ \Gamma $ into $ \Gamma_l $ and $ \Gamma_r $\;
					\nl \pop{$ \Gamma_l $, $ \Gamma_r $}\tcp*[r]{Make sure $ ab \in \Gamma_l\Gamma_r $ is non-crossing}
					\nl \ForEach{$ ab \in \Gamma_l\Gamma_r $}{
						\nl \compresspair{$ a $, $ b $} \;
					}
				}
			}
		}
	\end{algorithm}
\end{frame}

\begin{frame}
	\frametitle{Running example continued}
	
	We perform pair compression with $ \Gamma_l := \{a, \freshVar{a_2}, \freshVar{a_4}\} $ and $ \Gamma_r := \{b, c\} $:
	
	% After block compression, we obtained:
	
	\only<1-3|handout:1-3>{\[
		\overunderbraces{&&&&&&&\br{1}{\var{y}}&&&&&&\br{5}{\var{x}}}
		{&c&\ &a&\ &b&\ &c\ b&\ &\freshVar{a_4}&\ &c&\ &a&\ &b&\ &c\ b&\ &\freshVar{a_2}}
		{&&&\br{5}{\var{x}}&&&&&&&&&&\br{1}{\var{y}}}
	\]}\only<2-3|handout:2-3>{
	\stepdown{Set $ \var{y} := c\var{y} $ (because $ \varphi(\var{y}) $ starts with $ c \in \Gamma_r $)}
	}\only<2-|handout:2->{
		\[
		\overunderbraces{&&&&&&&\br{1}{\var{y}}&&&&&&\br{5}{\var{x}}}
		{&c&\ &a&\ &b\ c&\ &b&\ &\freshVar{a_4}&\ &c&\ &a&\ &b\ c&\ &b&\ &\freshVar{a_2}}
		{&&&\br{5}{\var{x}}&&&&&&&&&&\br{1}{\var{y}}}
		\]
	}\only<3-|handout:3->{
		\stepdown{Compress $ ab \rightsquigarrow \anotherFreshVar{d} $ ($ ab $ is guaranteed to be non-crossing)}\[
		\overunderbraces{&&&&&&&\br{1}{\var{y}}&&&&&&\br{5}{\var{x}}}
		{&c&\ &\anotherFreshVar{d}&\ &c&\ &b&\ &\freshVar{a_4}&\ &c&\ &\anotherFreshVar{d}&\ &c&\ &b&\ &\freshVar{a_2}}
		{&&&\br{5}{\var{x}}&&&&&&&&&&\br{1}{\var{y}}}
		\]
	}\only<4-|handout:4->{
		\stepdown{Compress $ \freshVar{a_4}c \rightsquigarrow \alertBlue{e} $ ($ \freshVar{a_4}c $ is guaranteed to be non-crossing)}\[
		\overunderbraces{&&&&&&&\br{1}{\var{y}}&&&&\br{5}{\var{x}}}
		{&c&\ &\anotherFreshVar{d}&\ &c&\ &b&\ &\alertBlue{e}&\ &\anotherFreshVar{d}&\ &c&\ &b&\ &\freshVar{a_2}}
		{&&&\br{5}{\var{x}}&&&&&&&&\br{1}{\var{y}}}
		\]\begin{center}
			\fbox{
				\parbox{0.75\textwidth}{
					There are no more occurrences of $ ab \in \Gamma_l \Gamma_r $ in $ U $ or $ V $.
				}
			}
		\end{center}
	}
\end{frame}

\section{Analysis of the recompression algorithm}

\begin{frame}
	\frametitle{Analysis of the algorithm -- overview}
	
	We analyze one iteration of the main loop (``one phase'') where $ U_1 = V_1 $ gets rewritten to $ U_4 = V_4 $.
	
	\begin{center}
		\begin{tikzpicture}
	% config
	
	\coordinate (blockwidth) at (2, 0);
	\coordinate (blockheight) at (0, 1.5);
	\coordinate (blockspacing) at (1, 0);
	\coordinate (stagearrlen) at (0, 1);
	
	% first block
	
	\coordinate (block1tl) at (0, 0);
	\coordinate (block1bl) at ($ (block1tl) - (blockheight) $);
	\coordinate (block1tr) at ($ (block1tl) + (blockwidth) $);
	\coordinate (block1br) at ($ (block1tr) - (blockheight) $);
	\coordinate (block1tc) at ($ (block1tl)!0.5!(block1tr) $);
	\coordinate (block1bc) at ($ (block1bl)!0.5!(block1br) $);
	\coordinate (block1rc) at ($ (block1tr)!0.5!(block1br) $);
	
	% second block
	
	\coordinate (block2tl) at ($ (block1tr) + (blockspacing) $);
	\coordinate (block2bl) at ($ (block2tl) - (blockheight) $);
	\coordinate (block2tr) at ($ (block2tl) + (blockwidth) $);
	\coordinate (block2br) at ($ (block2tr) - (blockheight) $);
	\coordinate (block2tc) at ($ (block2tl)!0.5!(block2tr) $);
	\coordinate (block2bc) at ($ (block2bl)!0.5!(block2br) $);
	\coordinate (block2lc) at ($ (block2tl)!0.5!(block2bl) $);
	\coordinate (block2rc) at ($ (block2tr)!0.5!(block2br) $);
	
	% third block
	
	\coordinate (block3tl) at ($ (block2tr) + (blockspacing) $);
	\coordinate (block3bl) at ($ (block3tl) - (blockheight) $);
	\coordinate (block3tr) at ($ (block3tl) + (blockwidth) $);
	\coordinate (block3br) at ($ (block3tr) - (blockheight) $);
	\coordinate (block3tc) at ($ (block3tl)!0.5!(block3tr) $);
	\coordinate (block3bc) at ($ (block3bl)!0.5!(block3br) $);
	\coordinate (block3lc) at ($ (block3tl)!0.5!(block3bl) $);
	\coordinate (block3rc) at ($ (block3tr)!0.5!(block3br) $);
	
	% fourth block
	
	\coordinate (block4tl) at ($ (block3tr) + (blockspacing) $);
	\coordinate (block4bl) at ($ (block4tl) - (blockheight) $);
	\coordinate (block4tr) at ($ (block4tl) + (blockwidth) $);
	\coordinate (block4br) at ($ (block4tr) - (blockheight) $);
	\coordinate (block4tc) at ($ (block4tl)!0.5!(block4tr) $);
	\coordinate (block4bc) at ($ (block4bl)!0.5!(block4br) $);
	\coordinate (block4lc) at ($ (block4tl)!0.5!(block4bl) $);
	
	% stages
	
	\coordinate (stage12arrend) at ($ (block1rc)!0.5!(block2lc) $);
	\coordinate (stage23arrend) at ($ (block2rc)!0.5!(block3lc) $);
	\coordinate (stage34arrend) at ($ (block3rc)!0.5!(block4lc) $);
	
	\coordinate (stage12arrbegin) at ($ (stage12arrend) + (stagearrlen) $);
	\coordinate (stage23arrbegin) at ($ (stage23arrend) + 1.5*(stagearrlen) $);
	\coordinate (stage34arrbegin) at ($ (stage34arrend) + (stagearrlen) $);
	
	% content of first block
	\draw ($ (block1tc)!0.33!(block1bc) $) node [rounded corners, draw=blue, fill=white] {$ U_1 = V_1 $};
	\draw ($ (block1tc)!0.75!(block1bc) $) node {Size: $ n_1 $};
	
	% content of second block
	\draw ($ (block2tc)!0.33!(block2bc) $) node [rounded corners, draw=blue, fill=white] {$ U_2 = V_2 $};
	\draw ($ (block2tc)!0.75!(block2bc) $) node {Size: $ n_2 $};
	
	% content of third block
	\draw ($ (block3tc)!0.33!(block3bc) $) node [rounded corners, draw=blue, fill=white] {$ U_3 = V_3 $};
	\draw ($ (block3tc)!0.75!(block3bc) $) node {Size: $ n_3 $};
	
	% content of fourth block
	\draw ($ (block4tc)!0.33!(block4bc) $) node [rounded corners, draw=blue, fill=white] {$ U_4 = V_4 $};
	\draw ($ (block4tc)!0.75!(block4bc) $) node {Size: $ n_4 $};
	
	% arrows between blocks
	\draw[->, thick] (block1rc) -- (block2lc);
	\draw[->, thick] (block2rc) -- (block3lc);
	\draw[->, thick] (block3rc) -- (block4lc);
	
	% stage arrows
	\draw[->, thick] (stage12arrbegin) -- (stage12arrend);
	\draw[->, thick] (stage23arrbegin) -- (stage23arrend);
	\draw[->, thick] (stage34arrbegin) -- (stage34arrend);
	
	% stage titles
	\node[anchor=south] at (stage12arrbegin) {Block compression};
	\node[anchor=south] at (stage23arrbegin) {Pair compression};
	\node[anchor=south] at (stage34arrbegin) {Pair compression};
	
	% draw block boxes
	\draw[thick] (block1tl) -- (block1tr) -- (block1br) -- (block1bl) -- cycle;
	\draw[thick] (block2tl) -- (block2tr) -- (block2br) -- (block2bl) -- cycle;
	\draw[thick] (block3tl) -- (block3tr) -- (block3br) -- (block3bl) -- cycle;
	\draw[thick] (block4tl) -- (block4tr) -- (block4br) -- (block4bl) -- cycle;
\end{tikzpicture}
	\end{center}

	\begin{columns}
		\column{0.45\textwidth}
		\textbf{Block compression}:
		
		{
		\setlength{\interspacetitleruled}{0pt}
		\setlength{\algotitleheightrule}{0pt}
		\begin{algorithm}[H]
			\nl \cutprefsuff{}\;
			\nl \ForEach{$ a \in \Gamma $}{
				\nl \compressblock{$ a $}\;
			}
		\end{algorithm}
		}
		\column{0.45\textwidth}
		\textbf{Pair compression}:
		
		{
		\setlength{\interspacetitleruled}{0pt}
		\setlength{\algotitleheightrule}{0pt}
		\begin{algorithm}[H]
			\nl Guess partition of $ \Gamma $ into $ \Gamma_l $ and $ \Gamma_r $\;
			\nl \pop{$ \Gamma_l $, $ \Gamma_r $}\;
			\nl \ForEach{$ ab \in \Gamma_l\Gamma_r $}{
				\nl \compresspair{$ a $, $ b $}\;
			}
		\end{algorithm}
		}
	\end{columns}
\end{frame}

\begin{frame}
	\frametitle{Compression of the solution}
	
	We analyze how many letters of the solution $ \varphi(U) $ are compressed after \textbf{Block compression} followed by (one iteration of) \textbf{Pair compression}.
	
	\begin{itemize}
		\item Let $ N := \left|\varphi(U)\right| $ and divide $ \varphi(U) $ into three-letter segments \[
			\varphi(U) = w_1 w_2 \dots w_{N/3}
		\] (w.l.o.g. $ N $ is divisible by $ 3 $).
		\item Consider $ w_i = abc $ and let all probabilities be over the space of partitions of $ \Gamma $.
		\item Let $ C_i : \Omega \rightarrow \{0, 1\} $ be a random variable indicating whether at least one letter in $ w_i $ is compressed.
		\item If $ a \neq b $ and $ b \neq c $, then \[
			\Pr[C_i = 1] = \Pr[ab \in \Gamma_l \Gamma_r \vee bc \in \Gamma_l \Gamma_r]
		\]
		\item We prove this in more detail. Suppose $ ab \in \Gamma_l \Gamma_r $. The case when $ ab $ is explicit or crossing is clear (\pop{$ \Gamma_l $, $ \Gamma_r $} makes $ ab $ explicit whereupon $ ab $ gets compressed).
	\end{itemize}
\end{frame}

\begin{frame}
	\frametitle{Compression of the solution}
	
	\begin{itemize}
		\item Similarly, but for a different reason:
		\item If $ ab $ is implicit, then \pop{$ \Gamma_l $, $ \Gamma_r $} also makes $ ab $ explicit whereupon $ ab $ will get compressed.
		\item This is because if all occurrences of $ ab $ are implicit, the solution $ \varphi $ is not length-minimal as we can get a shorter solution by defining $ \varphi' $ with $ \varphi'(\var{x}) $ set to $ \varphi(\var{x}) $ with every $ ab $ replaced by $ \varepsilon $ (see \cite[Lemma 6]{plandowsky_rytter:1998} and \cite[Lemma 2.4]{jez:2016}).
		\item This is a contradiction because it can be assumed that we are compressing a length-minimal solution.
		\item Hence, \begin{align*}
			\Pr[C_i = 1] &= \Pr[ab \in \Gamma_l \Gamma_r \vee bc \in \Gamma_l \Gamma_r] \\
			&= \Pr[ab \in \Gamma_l \Gamma_r] + \Pr[bc \in \Gamma_l \Gamma_r] \\
			&= 1/4 + 1/4 = 1/2
		\end{align*}
	\end{itemize}

\end{frame}

\begin{frame}
	\frametitle{Compression of the solution}

	\begin{itemize}
		\item Now suppose $ a = b $ or $ b = c $.
		\item Then $ abc $ intersects with some maximal block $ d^k $ in $ \varphi(U) $, for $ d \in \{a, b, c\} $ and $ k \ge 2 $.
		\item \textbf{Case 1}: The $ d^k $ occurrence is not implicit $ \Rightarrow $ \cutprefsuff{} makes $ d^k $ explicit $ \Rightarrow $ The $ d^k $ occurrence gets compressed.
		\item \textbf{Case 2}: Suppose the $ d^k $ occurrence is implicit.
		\item Then either some other occurrence of $ d^k $ causes $ d^k $ to get compressed, or all maximal $ d^k $ blocks are implicit.
		\item The latter implies that $ \varphi $ is not length-minimal because all occurrences of maximal $ d^k $ blocks can be deleted from the solution (ibid). This is a contradiction.
		\item Hence,
	\end{itemize}

	\begin{center}
		\fbox{
			\parbox{0.5\textwidth}{\begin{center}$ \Pr[C_i = 1] \ge 1/2 $\end{center}}
		}
	\end{center}
\end{frame}

\begin{frame}
	\frametitle{Compression of the solution}
	
	We now count the expected number of compressed $ w_i $ blocks: \begin{align*}
		\ExpVal\Bigg[\sum_{i=1}^{N/3} C_i\Bigg] &= \sum_{i=1}^{N/3} \ExpVal[C_i] \\
		&= \sum_{i=1}^{N/3} \Pr[C_i = 1] \\
		&\ge \sum_{i=1}^{N/3} 1/2 \ge \left\lfloor N/6\right\rfloor
	\end{align*}

	\pause

	$ \rightsquigarrow $ There exists a partition $ \Gamma = \Gamma_l \cup \Gamma_r $ such that $ \left\lfloor N/6\right\rfloor $ letters get compressed.
	
	\vspace{0.5em}
	
	$ \rightsquigarrow $ For appropriate nondeterministic choices, the recompression algorithm terminates after $ O(\log(N)) $ phases, where $ N $ is the size of the length-minimal solution.
\end{frame}

\begin{frame}
	\frametitle{Time complexity}
	
	\begin{itemize}
		\item We have just shown that $ O(\log(N)) $ phases are sufficient to ensure soundness and completeness.
		\item But what running time do we obtain in terms of the size of the input equation?
	\end{itemize}

	\pause
	
	\begin{theorem}[\cite{plandowski:1999}]
		Let $ N $ be the size of the length-minimal solution to a word equation of size $ n $. Then \[
			N \in O(2^{2^{\poly(n)}})
		\]
	\end{theorem}

	Consequently:
	\begin{theorem}
		The recompression algorithm runs in time exponential in the size of the given word equation.
	\end{theorem}
\end{frame}

\begin{frame}
	\frametitle{Space complexity analysis -- overview}
	
	\begin{itemize}
		\item We have shown the exponential time complexity of the recompression algorithm.
		\item But what about space complexity?
		\item Recall:
	\end{itemize}
	\begin{center}
		\begin{tikzpicture}
	% config
	
	\coordinate (blockwidth) at (2, 0);
	\coordinate (blockheight) at (0, 1.5);
	\coordinate (blockspacing) at (1, 0);
	\coordinate (stagearrlen) at (0, 1);
	
	% first block
	
	\coordinate (block1tl) at (0, 0);
	\coordinate (block1bl) at ($ (block1tl) - (blockheight) $);
	\coordinate (block1tr) at ($ (block1tl) + (blockwidth) $);
	\coordinate (block1br) at ($ (block1tr) - (blockheight) $);
	\coordinate (block1tc) at ($ (block1tl)!0.5!(block1tr) $);
	\coordinate (block1bc) at ($ (block1bl)!0.5!(block1br) $);
	\coordinate (block1rc) at ($ (block1tr)!0.5!(block1br) $);
	
	% second block
	
	\coordinate (block2tl) at ($ (block1tr) + (blockspacing) $);
	\coordinate (block2bl) at ($ (block2tl) - (blockheight) $);
	\coordinate (block2tr) at ($ (block2tl) + (blockwidth) $);
	\coordinate (block2br) at ($ (block2tr) - (blockheight) $);
	\coordinate (block2tc) at ($ (block2tl)!0.5!(block2tr) $);
	\coordinate (block2bc) at ($ (block2bl)!0.5!(block2br) $);
	\coordinate (block2lc) at ($ (block2tl)!0.5!(block2bl) $);
	\coordinate (block2rc) at ($ (block2tr)!0.5!(block2br) $);
	
	% third block
	
	\coordinate (block3tl) at ($ (block2tr) + (blockspacing) $);
	\coordinate (block3bl) at ($ (block3tl) - (blockheight) $);
	\coordinate (block3tr) at ($ (block3tl) + (blockwidth) $);
	\coordinate (block3br) at ($ (block3tr) - (blockheight) $);
	\coordinate (block3tc) at ($ (block3tl)!0.5!(block3tr) $);
	\coordinate (block3bc) at ($ (block3bl)!0.5!(block3br) $);
	\coordinate (block3lc) at ($ (block3tl)!0.5!(block3bl) $);
	\coordinate (block3rc) at ($ (block3tr)!0.5!(block3br) $);
	
	% fourth block
	
	\coordinate (block4tl) at ($ (block3tr) + (blockspacing) $);
	\coordinate (block4bl) at ($ (block4tl) - (blockheight) $);
	\coordinate (block4tr) at ($ (block4tl) + (blockwidth) $);
	\coordinate (block4br) at ($ (block4tr) - (blockheight) $);
	\coordinate (block4tc) at ($ (block4tl)!0.5!(block4tr) $);
	\coordinate (block4bc) at ($ (block4bl)!0.5!(block4br) $);
	\coordinate (block4lc) at ($ (block4tl)!0.5!(block4bl) $);
	
	% stages
	
	\coordinate (stage12arrend) at ($ (block1rc)!0.5!(block2lc) $);
	\coordinate (stage23arrend) at ($ (block2rc)!0.5!(block3lc) $);
	\coordinate (stage34arrend) at ($ (block3rc)!0.5!(block4lc) $);
	
	\coordinate (stage12arrbegin) at ($ (stage12arrend) + (stagearrlen) $);
	\coordinate (stage23arrbegin) at ($ (stage23arrend) + 1.5*(stagearrlen) $);
	\coordinate (stage34arrbegin) at ($ (stage34arrend) + (stagearrlen) $);
	
	% content of first block
	\draw ($ (block1tc)!0.33!(block1bc) $) node [rounded corners, draw=blue, fill=white] {$ U_1 = V_1 $};
	\draw ($ (block1tc)!0.75!(block1bc) $) node {Size: $ n_1 $};
	
	% content of second block
	\draw ($ (block2tc)!0.33!(block2bc) $) node [rounded corners, draw=blue, fill=white] {$ U_2 = V_2 $};
	\draw ($ (block2tc)!0.75!(block2bc) $) node {Size: $ n_2 $};
	
	% content of third block
	\draw ($ (block3tc)!0.33!(block3bc) $) node [rounded corners, draw=blue, fill=white] {$ U_3 = V_3 $};
	\draw ($ (block3tc)!0.75!(block3bc) $) node {Size: $ n_3 $};
	
	% content of fourth block
	\draw ($ (block4tc)!0.33!(block4bc) $) node [rounded corners, draw=blue, fill=white] {$ U_4 = V_4 $};
	\draw ($ (block4tc)!0.75!(block4bc) $) node {Size: $ n_4 $};
	
	% arrows between blocks
	\draw[->, thick] (block1rc) -- (block2lc);
	\draw[->, thick] (block2rc) -- (block3lc);
	\draw[->, thick] (block3rc) -- (block4lc);
	
	% stage arrows
	\draw[->, thick] (stage12arrbegin) -- (stage12arrend);
	\draw[->, thick] (stage23arrbegin) -- (stage23arrend);
	\draw[->, thick] (stage34arrbegin) -- (stage34arrend);
	
	% stage titles
	\node[anchor=south] at (stage12arrbegin) {Block compression};
	\node[anchor=south] at (stage23arrbegin) {Pair compression};
	\node[anchor=south] at (stage34arrbegin) {Pair compression};
	
	% draw block boxes
	\draw[thick] (block1tl) -- (block1tr) -- (block1br) -- (block1bl) -- cycle;
	\draw[thick] (block2tl) -- (block2tr) -- (block2br) -- (block2bl) -- cycle;
	\draw[thick] (block3tl) -- (block3tr) -- (block3br) -- (block3bl) -- cycle;
	\draw[thick] (block4tl) -- (block4tr) -- (block4br) -- (block4bl) -- cycle;
\end{tikzpicture}
	\end{center}
	\begin{itemize}
		\item We show that at any point during the algorithm the size of the equation is linear, i.e., at most $ kn $ for some constant $ k $.
		\item For this, it suffices to show that $ n_1 \le kn $ implies $ n_4 \le kn $.
	\end{itemize}
\end{frame}

\begin{frame}
	\frametitle{Preliminary observations}
	
	For the sake of convenience, we introduce the notion of a \textit{symbol}.
	
	\begin{center}
		\begin{tikzpicture}[
			grow=down,
			edge from parent/.style = {draw, -latex},
			sibling distance        = 13em,
			level distance          = 4em,
			]
			\node {Symbol}
			child {node {$ a \in \Gamma $}}
			child {node {Block $ a^l $ popped from some variable $ \var{x} $}};
			% child {node {Variable $ \var{x} $}};
		\end{tikzpicture}
	\end{center}
	
	\pause
	
	\begin{itemize}
		\item The algorithm never changes the number of variable occurrences in the equation.
		\item Let that number be $ m $.
		\item Pair compression introduces at most $ 2m $ symbols.
		\item Block compression also introduces at most $ 2m $ symbols.
	\end{itemize}
	
	$ \rightsquigarrow $ One phase introduces at most $ 6m $ symbols into the equation.
\end{frame}

\begin{frame}
	\frametitle{Compression of the equation}
	
	We analyze how many letters in the current equation are compressed after the second execution of \textbf{Pair compression}.
	
	\begin{center}
		\begin{tikzpicture}
	% config
	
	\coordinate (blockwidth) at (2, 0);
	\coordinate (blockheight) at (0, 1.5);
	\coordinate (blockspacing) at (1, 0);
	\coordinate (stagearrlen) at (0, 1);
	
	% first block
	
	\coordinate (block1tl) at (0, 0);
	\coordinate (block1bl) at ($ (block1tl) - (blockheight) $);
	\coordinate (block1tr) at ($ (block1tl) + (blockwidth) $);
	\coordinate (block1br) at ($ (block1tr) - (blockheight) $);
	\coordinate (block1tc) at ($ (block1tl)!0.5!(block1tr) $);
	\coordinate (block1bc) at ($ (block1bl)!0.5!(block1br) $);
	\coordinate (block1rc) at ($ (block1tr)!0.5!(block1br) $);
	
	% second block
	
	\coordinate (block2tl) at ($ (block1tr) + (blockspacing) $);
	\coordinate (block2bl) at ($ (block2tl) - (blockheight) $);
	\coordinate (block2tr) at ($ (block2tl) + (blockwidth) $);
	\coordinate (block2br) at ($ (block2tr) - (blockheight) $);
	\coordinate (block2tc) at ($ (block2tl)!0.5!(block2tr) $);
	\coordinate (block2bc) at ($ (block2bl)!0.5!(block2br) $);
	\coordinate (block2lc) at ($ (block2tl)!0.5!(block2bl) $);
	\coordinate (block2rc) at ($ (block2tr)!0.5!(block2br) $);
	
	% third block
	
	\coordinate (block3tl) at ($ (block2tr) + (blockspacing) $);
	\coordinate (block3bl) at ($ (block3tl) - (blockheight) $);
	\coordinate (block3tr) at ($ (block3tl) + (blockwidth) $);
	\coordinate (block3br) at ($ (block3tr) - (blockheight) $);
	\coordinate (block3tc) at ($ (block3tl)!0.5!(block3tr) $);
	\coordinate (block3bc) at ($ (block3bl)!0.5!(block3br) $);
	\coordinate (block3lc) at ($ (block3tl)!0.5!(block3bl) $);
	\coordinate (block3rc) at ($ (block3tr)!0.5!(block3br) $);
	
	% fourth block
	
	\coordinate (block4tl) at ($ (block3tr) + (blockspacing) $);
	\coordinate (block4bl) at ($ (block4tl) - (blockheight) $);
	\coordinate (block4tr) at ($ (block4tl) + (blockwidth) $);
	\coordinate (block4br) at ($ (block4tr) - (blockheight) $);
	\coordinate (block4tc) at ($ (block4tl)!0.5!(block4tr) $);
	\coordinate (block4bc) at ($ (block4bl)!0.5!(block4br) $);
	\coordinate (block4lc) at ($ (block4tl)!0.5!(block4bl) $);
	
	% stages
	
	\coordinate (stage12arrend) at ($ (block1rc)!0.5!(block2lc) $);
	\coordinate (stage23arrend) at ($ (block2rc)!0.5!(block3lc) $);
	\coordinate (stage34arrend) at ($ (block3rc)!0.5!(block4lc) $);
	
	\coordinate (stage12arrbegin) at ($ (stage12arrend) + (stagearrlen) $);
	\coordinate (stage23arrbegin) at ($ (stage23arrend) + 1.5*(stagearrlen) $);
	\coordinate (stage34arrbegin) at ($ (stage34arrend) + (stagearrlen) $);
	
	% content of first block
	\draw ($ (block1tc)!0.33!(block1bc) $) node [rounded corners, draw=blue, fill=white] {$ U_1 = V_1 $};
	\draw ($ (block1tc)!0.75!(block1bc) $) node {Size: $ n_1 $};
	
	% content of second block
	\draw ($ (block2tc)!0.33!(block2bc) $) node [rounded corners, draw=blue, fill=white] {$ U_2 = V_2 $};
	\draw ($ (block2tc)!0.75!(block2bc) $) node {Size: $ n_2 $};
	
	% content of third block
	\draw ($ (block3tc)!0.33!(block3bc) $) node [rounded corners, draw=blue, fill=white] {$ U_3 = V_3 $};
	\draw ($ (block3tc)!0.75!(block3bc) $) node {Size: $ n_3 $};
	
	% content of fourth block
	\draw ($ (block4tc)!0.33!(block4bc) $) node [rounded corners, draw=blue, fill=white] {$ U_4 = V_4 $};
	\draw ($ (block4tc)!0.75!(block4bc) $) node {Size: $ n_4 $};
	
	% arrows between blocks
	\draw[->, thick] (block1rc) -- (block2lc);
	\draw[->, thick] (block2rc) -- (block3lc);
	\draw[->, thick] (block3rc) -- (block4lc);
	
	% stage arrows
	\draw[->, thick] (stage12arrbegin) -- (stage12arrend);
	\draw[->, thick] (stage23arrbegin) -- (stage23arrend);
	\draw[->, thick] (stage34arrbegin) -- (stage34arrend);
	
	% stage titles
	\node[anchor=south] at (stage12arrbegin) {Block compression};
	\node[anchor=south] at (stage23arrbegin) {Pair compression};
	\node[anchor=south] at (stage34arrbegin) {Pair compression};
	
	% draw block boxes
	\draw[thick] (block1tl) -- (block1tr) -- (block1br) -- (block1bl) -- cycle;
	\draw[thick] (block2tl) -- (block2tr) -- (block2br) -- (block2bl) -- cycle;
	\draw[thick] (block3tl) -- (block3tr) -- (block3br) -- (block3bl) -- cycle;
	\draw[thick] (block4tl) -- (block4tr) -- (block4br) -- (block4bl) -- cycle;
\end{tikzpicture}
	\end{center}

	I.e., what is the relationship between $ n_3 $ and $ n_4 $?
	
	\pause
	
	\begin{itemize}
		\item Clearly, $ U_2 = V_2 $ doesn't contain any $ a^k $ block for $ k \ge 2 $.
		\item Hence, $ U_3 = V_3 $ doesn't contain any $ a^k $ block for $ k \ge 3 $ (note that here we implicitly rely on the fact that no character $ a \in \Gamma $ can be popped both to the right of some variable, as well as to the left of it, see the implementation of $ \pop $ and note that $ \Gamma_l \cap \Gamma_r = \varnothing $ is crucial).
	\end{itemize}
\end{frame}

\begin{frame}
	\frametitle{Compression of the equation}
	
	\begin{center}
		\begin{tikzpicture}
	% config
	
	\coordinate (blockwidth) at (2, 0);
	\coordinate (blockheight) at (0, 1.5);
	\coordinate (blockspacing) at (1, 0);
	\coordinate (stagearrlen) at (0, 1);
	
	% first block
	
	\coordinate (block1tl) at (0, 0);
	\coordinate (block1bl) at ($ (block1tl) - (blockheight) $);
	\coordinate (block1tr) at ($ (block1tl) + (blockwidth) $);
	\coordinate (block1br) at ($ (block1tr) - (blockheight) $);
	\coordinate (block1tc) at ($ (block1tl)!0.5!(block1tr) $);
	\coordinate (block1bc) at ($ (block1bl)!0.5!(block1br) $);
	\coordinate (block1rc) at ($ (block1tr)!0.5!(block1br) $);
	
	% second block
	
	\coordinate (block2tl) at ($ (block1tr) + (blockspacing) $);
	\coordinate (block2bl) at ($ (block2tl) - (blockheight) $);
	\coordinate (block2tr) at ($ (block2tl) + (blockwidth) $);
	\coordinate (block2br) at ($ (block2tr) - (blockheight) $);
	\coordinate (block2tc) at ($ (block2tl)!0.5!(block2tr) $);
	\coordinate (block2bc) at ($ (block2bl)!0.5!(block2br) $);
	\coordinate (block2lc) at ($ (block2tl)!0.5!(block2bl) $);
	\coordinate (block2rc) at ($ (block2tr)!0.5!(block2br) $);
	
	% third block
	
	\coordinate (block3tl) at ($ (block2tr) + (blockspacing) $);
	\coordinate (block3bl) at ($ (block3tl) - (blockheight) $);
	\coordinate (block3tr) at ($ (block3tl) + (blockwidth) $);
	\coordinate (block3br) at ($ (block3tr) - (blockheight) $);
	\coordinate (block3tc) at ($ (block3tl)!0.5!(block3tr) $);
	\coordinate (block3bc) at ($ (block3bl)!0.5!(block3br) $);
	\coordinate (block3lc) at ($ (block3tl)!0.5!(block3bl) $);
	\coordinate (block3rc) at ($ (block3tr)!0.5!(block3br) $);
	
	% fourth block
	
	\coordinate (block4tl) at ($ (block3tr) + (blockspacing) $);
	\coordinate (block4bl) at ($ (block4tl) - (blockheight) $);
	\coordinate (block4tr) at ($ (block4tl) + (blockwidth) $);
	\coordinate (block4br) at ($ (block4tr) - (blockheight) $);
	\coordinate (block4tc) at ($ (block4tl)!0.5!(block4tr) $);
	\coordinate (block4bc) at ($ (block4bl)!0.5!(block4br) $);
	\coordinate (block4lc) at ($ (block4tl)!0.5!(block4bl) $);
	
	% stages
	
	\coordinate (stage12arrend) at ($ (block1rc)!0.5!(block2lc) $);
	\coordinate (stage23arrend) at ($ (block2rc)!0.5!(block3lc) $);
	\coordinate (stage34arrend) at ($ (block3rc)!0.5!(block4lc) $);
	
	\coordinate (stage12arrbegin) at ($ (stage12arrend) + (stagearrlen) $);
	\coordinate (stage23arrbegin) at ($ (stage23arrend) + 1.5*(stagearrlen) $);
	\coordinate (stage34arrbegin) at ($ (stage34arrend) + (stagearrlen) $);
	
	% content of first block
	\draw ($ (block1tc)!0.33!(block1bc) $) node [rounded corners, draw=blue, fill=white] {$ U_1 = V_1 $};
	\draw ($ (block1tc)!0.75!(block1bc) $) node {Size: $ n_1 $};
	
	% content of second block
	\draw ($ (block2tc)!0.33!(block2bc) $) node [rounded corners, draw=blue, fill=white] {$ U_2 = V_2 $};
	\draw ($ (block2tc)!0.75!(block2bc) $) node {Size: $ n_2 $};
	
	% content of third block
	\draw ($ (block3tc)!0.33!(block3bc) $) node [rounded corners, draw=blue, fill=white] {$ U_3 = V_3 $};
	\draw ($ (block3tc)!0.75!(block3bc) $) node {Size: $ n_3 $};
	
	% content of fourth block
	\draw ($ (block4tc)!0.33!(block4bc) $) node [rounded corners, draw=blue, fill=white] {$ U_4 = V_4 $};
	\draw ($ (block4tc)!0.75!(block4bc) $) node {Size: $ n_4 $};
	
	% arrows between blocks
	\draw[->, thick] (block1rc) -- (block2lc);
	\draw[->, thick] (block2rc) -- (block3lc);
	\draw[->, thick] (block3rc) -- (block4lc);
	
	% stage arrows
	\draw[->, thick] (stage12arrbegin) -- (stage12arrend);
	\draw[->, thick] (stage23arrbegin) -- (stage23arrend);
	\draw[->, thick] (stage34arrbegin) -- (stage34arrend);
	
	% stage titles
	\node[anchor=south] at (stage12arrbegin) {Block compression};
	\node[anchor=south] at (stage23arrbegin) {Pair compression};
	\node[anchor=south] at (stage34arrbegin) {Pair compression};
	
	% draw block boxes
	\draw[thick] (block1tl) -- (block1tr) -- (block1br) -- (block1bl) -- cycle;
	\draw[thick] (block2tl) -- (block2tr) -- (block2br) -- (block2bl) -- cycle;
	\draw[thick] (block3tl) -- (block3tr) -- (block3br) -- (block3bl) -- cycle;
	\draw[thick] (block4tl) -- (block4tr) -- (block4br) -- (block4bl) -- cycle;
\end{tikzpicture}
	\end{center}
	
	Write \begin{center}
		\begin{tabular}{c c c}
			$ U_3 $ & $ = $ & $ V_3 $ \\
			\rotatebox[origin=c]{90}{$ = $} && \rotatebox[origin=c]{90}{$ = $} \\
			$ u_1 x_1 u_2 x_2 u_3 \dots u_l x_l u_{l+1} $ & $ = $ & $ u_{l+2} x_{l+1} u_{l+3} \dots u_{m+1} x_m u_{m+2} $
		\end{tabular}
	\end{center} where $ u_i \in \Gamma^* $ (w.l.o.g., $ u_i $ is defined for all $ i \in \{1, \dots, m+2\} $).

	\begin{itemize}
		\item Clearly, $ \sum_{i=1}^{m+2} \left|u_i\right| = \left| u_1 u_2 \dots u_{m+2} \right| \ge n_3 - m $.
		\item Every $ u_i $ can be split into $ \left\lfloor {\left|u_i\right|}/3 \right\rfloor \ge {\left|u_i\right|}/3 - 2/3 $ three-letter segments.
	\end{itemize}
\end{frame}

\begin{frame}
	\frametitle{Compression of the equation}
	
	$ \rightsquigarrow $ The overall number of three-letter segments we can find in $ U_3 = V_3 $ is at least \begin{align*}
		\sum_{i=1}^{m+2} \Bigg(\frac{\left|u_i\right|}{3} - \frac{2}{3}\Bigg) &= \frac{\sum_{i=1}^{m+2} \left|u_i\right|}{3} - \frac{2(m+2)}{3} \\
		&\ge \frac{n_3 - m}{3} - \frac{2m+4}{3} \\
		&= \frac{n_3 - 3m - 4}{3}
	\end{align*}
\end{frame}

\begin{frame}
	\frametitle{Compression of the equation}
	
	\begin{itemize}
		\item Let $ w_j = abc $ be a three-letter segment of an arbitrary $ u_i $.
		\item Let $ C_j : \Omega \rightarrow \{0, 1\} $ be a random variable indicating whether at least one letter in $ w_j $ is compressed.
		\item Since $ U_3 = V_3 $ cannot contain any $ a^k $ block for any $ a \in \Gamma $ and $ k \ge 3 $ (see previous slides), it follows that $ a \neq b $ or $ b \neq c $.
		\item In any event, \[
			\Pr[C_j = 1] \ge 1/4
		\]
		\item Using the bound of the previous slide, we obtain that the expected number of compressed $ w_j $ blocks in the entire $ U_3 = V_3 $ equation is at least \begin{align*}
			\ExpVal\Bigg[\sum_{j=1}^{(n_3-3m-4)/3} C_j\Bigg] &= \sum_{j=1}^{(n_3-3m-4)/3} \ExpVal[C_j] = \sum_{j=1}^{(n_3-3m-4)/3} \Pr[C_j = 1] \\
			&\ge \sum_{i=1}^{(n_3-3m-4)/3} 1/4 = \frac{n_3 - 3m - 4}{12}
		\end{align*}
	\end{itemize}
\end{frame}

\begin{frame}
	\frametitle{Size of the equation}
	
	\begin{center}
		\begin{tikzpicture}
	% config
	
	\coordinate (blockwidth) at (2, 0);
	\coordinate (blockheight) at (0, 1.5);
	\coordinate (blockspacing) at (1, 0);
	\coordinate (stagearrlen) at (0, 1);
	
	% first block
	
	\coordinate (block1tl) at (0, 0);
	\coordinate (block1bl) at ($ (block1tl) - (blockheight) $);
	\coordinate (block1tr) at ($ (block1tl) + (blockwidth) $);
	\coordinate (block1br) at ($ (block1tr) - (blockheight) $);
	\coordinate (block1tc) at ($ (block1tl)!0.5!(block1tr) $);
	\coordinate (block1bc) at ($ (block1bl)!0.5!(block1br) $);
	\coordinate (block1rc) at ($ (block1tr)!0.5!(block1br) $);
	
	% second block
	
	\coordinate (block2tl) at ($ (block1tr) + (blockspacing) $);
	\coordinate (block2bl) at ($ (block2tl) - (blockheight) $);
	\coordinate (block2tr) at ($ (block2tl) + (blockwidth) $);
	\coordinate (block2br) at ($ (block2tr) - (blockheight) $);
	\coordinate (block2tc) at ($ (block2tl)!0.5!(block2tr) $);
	\coordinate (block2bc) at ($ (block2bl)!0.5!(block2br) $);
	\coordinate (block2lc) at ($ (block2tl)!0.5!(block2bl) $);
	\coordinate (block2rc) at ($ (block2tr)!0.5!(block2br) $);
	
	% third block
	
	\coordinate (block3tl) at ($ (block2tr) + (blockspacing) $);
	\coordinate (block3bl) at ($ (block3tl) - (blockheight) $);
	\coordinate (block3tr) at ($ (block3tl) + (blockwidth) $);
	\coordinate (block3br) at ($ (block3tr) - (blockheight) $);
	\coordinate (block3tc) at ($ (block3tl)!0.5!(block3tr) $);
	\coordinate (block3bc) at ($ (block3bl)!0.5!(block3br) $);
	\coordinate (block3lc) at ($ (block3tl)!0.5!(block3bl) $);
	\coordinate (block3rc) at ($ (block3tr)!0.5!(block3br) $);
	
	% fourth block
	
	\coordinate (block4tl) at ($ (block3tr) + (blockspacing) $);
	\coordinate (block4bl) at ($ (block4tl) - (blockheight) $);
	\coordinate (block4tr) at ($ (block4tl) + (blockwidth) $);
	\coordinate (block4br) at ($ (block4tr) - (blockheight) $);
	\coordinate (block4tc) at ($ (block4tl)!0.5!(block4tr) $);
	\coordinate (block4bc) at ($ (block4bl)!0.5!(block4br) $);
	\coordinate (block4lc) at ($ (block4tl)!0.5!(block4bl) $);
	
	% stages
	
	\coordinate (stage12arrend) at ($ (block1rc)!0.5!(block2lc) $);
	\coordinate (stage23arrend) at ($ (block2rc)!0.5!(block3lc) $);
	\coordinate (stage34arrend) at ($ (block3rc)!0.5!(block4lc) $);
	
	\coordinate (stage12arrbegin) at ($ (stage12arrend) + (stagearrlen) $);
	\coordinate (stage23arrbegin) at ($ (stage23arrend) + 1.5*(stagearrlen) $);
	\coordinate (stage34arrbegin) at ($ (stage34arrend) + (stagearrlen) $);
	
	% content of first block
	\draw ($ (block1tc)!0.33!(block1bc) $) node [rounded corners, draw=blue, fill=white] {$ U_1 = V_1 $};
	\draw ($ (block1tc)!0.75!(block1bc) $) node {Size: $ n_1 $};
	
	% content of second block
	\draw ($ (block2tc)!0.33!(block2bc) $) node [rounded corners, draw=blue, fill=white] {$ U_2 = V_2 $};
	\draw ($ (block2tc)!0.75!(block2bc) $) node {Size: $ n_2 $};
	
	% content of third block
	\draw ($ (block3tc)!0.33!(block3bc) $) node [rounded corners, draw=blue, fill=white] {$ U_3 = V_3 $};
	\draw ($ (block3tc)!0.75!(block3bc) $) node {Size: $ n_3 $};
	
	% content of fourth block
	\draw ($ (block4tc)!0.33!(block4bc) $) node [rounded corners, draw=blue, fill=white] {$ U_4 = V_4 $};
	\draw ($ (block4tc)!0.75!(block4bc) $) node {Size: $ n_4 $};
	
	% arrows between blocks
	\draw[->, thick] (block1rc) -- (block2lc);
	\draw[->, thick] (block2rc) -- (block3lc);
	\draw[->, thick] (block3rc) -- (block4lc);
	
	% stage arrows
	\draw[->, thick] (stage12arrbegin) -- (stage12arrend);
	\draw[->, thick] (stage23arrbegin) -- (stage23arrend);
	\draw[->, thick] (stage34arrbegin) -- (stage34arrend);
	
	% stage titles
	\node[anchor=south] at (stage12arrbegin) {Block compression};
	\node[anchor=south] at (stage23arrbegin) {Pair compression};
	\node[anchor=south] at (stage34arrbegin) {Pair compression};
	
	% draw block boxes
	\draw[thick] (block1tl) -- (block1tr) -- (block1br) -- (block1bl) -- cycle;
	\draw[thick] (block2tl) -- (block2tr) -- (block2br) -- (block2bl) -- cycle;
	\draw[thick] (block3tl) -- (block3tr) -- (block3br) -- (block3bl) -- cycle;
	\draw[thick] (block4tl) -- (block4tr) -- (block4br) -- (block4bl) -- cycle;
\end{tikzpicture}
	\end{center}

	\begin{itemize}
		\item Recall that our overall goal was to show that $ n_1 \le kn $ implies $ n_4 \le kn $ for an appropriate constant $ k $.
		\item Since \textbf{Pair compression} introduces at most $ 2m $ new symbols into the equation (see previous slides), and at least $ (n_3 - 3m - 4)/12 $ letters get compressed in the second iteration of \textbf{Pair compression} (for an appropriate choice of $ \Gamma_l $, $ \Gamma_r $), \[
			n_4 \le n_3 + 2m - \frac{n_3 - 3m - 4}{12}
		\]
	\end{itemize}
\end{frame}

\begin{frame}
	\frametitle{Size of the equation}
	
	\begin{itemize}
		\item Hence, \begin{align*}
			n_4 &\le n_3 + 2m - \frac{n_3 - 3m - 4}{12} = \frac{12 n_3 + 24 m - n_3 + 3m + 4}{12} \\
			&= \frac{11 n_3 + 27 m + 4}{12} \overset{n_3 \le n_1 + 4m}{\le} \frac{11 n_1 + 44 m + 27 m + 4}{12} \\
			&= \frac{11 n_1 + 71 m + 4}{12} \overset{n_1 \le kn}{\le} \frac{11 kn + 71 m + 4}{12} \overset{m \le n}{\le} \frac{11 kn + 71 n + 4}{12} \\
			&\overset{n \ge 4}{\le} \frac{11 kn + 72 n}{12} = \Bigg(\frac{11}{12}k + 6\Bigg)n \overset{\text{Goal}}{\le} kn
		\end{align*}
		\item Indeed, the last inequality is equivalent to $ (11/12)k + 6 \le k $ and thus holds for all $ k \ge 12 \cdot 6 = 72 $.
		\item In the calculation we used the fact that $ n_3 \le n_1 + 4m $; this is because \textit{Pair compression} and \textit{Block compression} each can introduce at most $ 2m $ new symbols into the equation.
	\end{itemize}
\end{frame}

\begin{frame}
	\frametitle{Size of the equation and space complexity}
	
	\begin{itemize}
		\item We conclude that for any satisfiable word equation $ U = V $ there exist nondeterministic choices leading to an accepting run of the recompression algorithm with all intermediate equations having size at most $ 72n \in O(n) $.
		\item This bound is on the amount of symbols and variables, not bits.
		\item Clearly, every variable $ \var{x} \in \X $ and letter $ a \in \Gamma $ can be encoded using $ O(\log(n)) $ bits.
	\end{itemize}

	$ \rightsquigarrow $ The symbols and variables appearing in the equation can be encoded using $ O(n \log(n)) $ bits in total.
\end{frame}

\begin{frame}
	\frametitle{Space complexity}
	
	\begin{itemize}
		\item But what about the space required to store blocks $ a^l $ that got popped from some variable $ \var{x} $?
		\item To properly bound that space, we need the following result by Ko{\'{s}}cielski and Pacholski:
	\end{itemize}

	\begin{lemma}[Exponent of periodicity bound \cite{koscielski_pacholski:1996}]
		Let $ \varphi $ be a length-minimal solution to a word equation $ U = V $ of size $ n $. Suppose $ \varphi(U) $ has a substring $ w^l $ for some $ w \in \Gamma^+ $. Then $ l \in 2^{O(n)} $.
	\end{lemma}

	$ \rightsquigarrow $ Every block $ a^l $ popped from some variable by \cutprefsuff{} can be encoded using $ O(n) $ bits using binary encoding.
	
	$ \rightsquigarrow $ The space complexity of the recompression algorithm is $ O(n^2) $ (bits).
\end{frame}

\section{Conclusion}

\begin{frame}
	\frametitle{Conclusion and references}
	
	Overall, we have shown:
	
	\begin{theorem}
		The recompression algorithm for word equations runs in nondeterministic $ O(\poly(n)\log(N)) $ time and uses $ O(n^2) $ space, where $ N $ is the size of the length-minimal solution. In terms of $ n $, the time complexity is $ O(2^{\poly(n)}) $.
	\end{theorem}

	\begin{corollary}
		If every word equation has a solution of exponential size, then word equations are in $ \NP $.
	\end{corollary}

	This motivates the following question:
	
	\begin{openproblem}{Open problem}
		Does every word equation have a solution of exponential size?
	\end{openproblem}

	\textbf{Best known lower bound}: $ 2^{o(n)} $
\end{frame}

\begin{frame}[allowframebreaks]
\frametitle{References}
\bibliographystyle{amsalpha}
\bibliography{we_recompression}
\end{frame}

\end{document}